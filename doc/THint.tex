\documentclass[english,twoside, openright]{report}
\usepackage[utf8]{inputenc}
\usepackage{babel}
\usepackage{physics}
\usepackage{xcolor}
\usepackage{amsmath}
\usepackage{amsfonts}
\usepackage{lineno,hyperref}
\usepackage{hyperref}

\newcommand{\eigket}[1]{
  \left| \left. \varphi^{\omega_{#1},J_{#1}}_{n_{#1},L_{#1}};\;\Lambda_{#1}\right.\right\rangle
}


\newcommand{\eigbra}[1]{
  \left\langle \left. \varphi^{\omega_{#1},J_{#1}}_{n_{#1},L_{#1}};\;\Lambda_{#1}\right.\right|
}

\newcommand{\eigval}[1]{
  \text{E}^{\omega_{#1},J_{#1}}_{n_{#1},L_{#1}}
}

\newcommand{\lamblock}[1]{
  \left\{
    \begin{matrix}
      \hat{\omega}_{#1} , \hat{J}_{#1} \\
      \hat{n}_{#1} , \hat{L}_{#1} 
    \end{matrix}
  \right\}_{\Lambda_{#1}} 
}

\newcommand{\coef}[1]{
  \mathcal{C}^{\Lambda_{#1}}_{\hat{\omega}_{#1},\hat{J}_{#1};\hat{n}_{#1},\hat{L}_{#1}}
}


\newcommand{\coefm}[1]{
  \mathcal{C}^{\Lambda_{#1}}_{\varphi_{#1}}
}

\newcommand{\pqket}[1]{
  \left| \left. \hat{\omega}_{#1} \hat{J}_{#1},\hat{n}_{#1}\hat{L}_{#1} ; \Lambda_{#1} \right.\right\rangle
}

\newcommand{\pqbra}[1]{
  \left\langle \left. \hat{\omega}_{#1} \hat{J}_{#1},\hat{n}_{#1}\hat{L}_{#1} ; \Lambda_{#1} \right.\right|
}

\newcommand{\ppmat}[1]{
  \left[\;
    \underline{\underline{
        #1
      }}
  \;\right]
}

\author{Jamil KR}
\title{$U_p(4)\; \times\; U_q(4)$. Intensity lines.}

\begin{document}
\maketitle

\begin{equation}
  \eigket{1} \; \eigbra{2} \; \lamblock{1} \pqket{1} \pqbra{2}\; \ppmat{C}
\end{equation}

\section{Eigensystem}
Hamiltonian:
\begin{equation}
  \hat{H} \; = \; \hat{H}_{\text{mol}} \; + \; \hat{H}_{\text{cage}} \;+\; \hat{H}_{\text{int}}
\end{equation}

Eigenvalues problem:
\begin{align}
  \hat{H}\ket{\varphi}&=E\ket{\varphi} \nonumber \\
  \hat{H}\eigket{} &= \eigval{}\eigket{} \\
  \eigket{} & = \sum\limits_{\tiny{\lamblock{}}} \coef{} \pqket{} \nonumber
\end{align}

The quantum numbers of the eigenstates are assigned according to the
maximum coefficient over the basis. The summation is over all the
states that belong to the para/ortho $\Lambda$-block.

\section{$\hat{d}^{(1)}$  dipolar operator}
Let $\hat{d}^{(1)}$ be a dipolar operator which doesn't mix para and ortho states.

\begin{align}
  & \eigbra{1}\left|\hat{d}^{(1)}\right| \eigket{2}  = \\
  = & \left[\sum\limits_{\tiny{\lamblock{1}}}
      \coef{1}\pqbra{1}
      \right]
      \left|\hat{d}^{(1)}\right|
      \left[\sum\limits_{\tiny{\lamblock{2}}}
      \coef{2}\pqket{2}
      \right] \nonumber \\
  = & \sum\limits_{\tiny{\lamblock{1}}} \sum\limits_{\tiny{\lamblock{2}}}
      \coef{1}\coef{2}\pqbra{1} \left|\hat{d}^{(1)}\right| \pqket{2} \nonumber \\
  = & \ppmat{\coefm{1}}^{T} \ppmat{d^{(1)}_{\Lambda_1,\Lambda_2}} \ppmat{\coefm{2}}
\end{align}

where $\ppmat{\coefm{}}$ is a
$\left(\text{dim\_}\Lambda\times 1\right)$ matrix. The dimension of
$\ppmat{\coefm{1}}^{T}$ is
$\left(1\times\text{dim\_}\Lambda_1\right)$, of
$\ppmat{d^{(1)}_{\Lambda_1,\Lambda_2}}$ is
$\left(\text{dim\_}\Lambda_1\times\text{dim\_}\Lambda_2\right)$, and
$\ppmat{\coefm{2}}$, $\left(\text{dim\_}\Lambda_2\times 1\right)$.

Now we need to build all the $\ppmat{d^{(1)}_{\Lambda_1,\Lambda_2}}$
for para and ortho cases when it's needed.

Steps:
\begin{enumerate}
\item Read experimental data and allocate the needed matrices.
\item Solve the eigensystem and compute the eigenstates
\item ... 
\end{enumerate}

\end{document}